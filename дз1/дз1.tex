\documentclass[12pt, a4paper]{article}
\usepackage{gset}

\begin{document}
	\maintitle{Матан. Дз 1}
	\B{1. Найти взаимнопростые m, n такие, что}
	$\frac{m}{n} = 5,(2023)$\\\\
	Пусть $\frac{m}{n} = x$; $x = 5,(2023)$\\
	\\
	$10000x = 52023,(2023)$\\
	\\
	$9999x = 52023,(2023) - 5,(2023)$\\
	\\
	$x = \frac{52018}{9999}$\\\\
	НОД(52018, 9999) = 0, значит 52018 и 9999 взаимнопростые числа\\\\
	\answer{m = 52018, n = 9999} \bspace
	\B{2. Вычислить суммы}\\
	\\
	a) $A_n = \frac{1}{1 * 5} + \frac{1}{5 * 9} + \frac{1}{9 * 13} + \cdots + \frac{1}{(4n - 3) * (4n + 1)} $\\
	\\
	Разложим каждый элемент последовательности используя равенство \\\\
	$\frac{1}{k(k + 4)} = \frac{1}{4}(\frac{4}{k(k + 4)}) = \frac{1}{4}(\frac{4 + k - k}{k(k + 4)}) = \frac{1}{4}(\frac{k + 4}{k(k + 4)} - \frac{k}{k(k + 4)}) = \frac{1}{4}(\frac{1}{k} - \frac{1}{k + 4})$\\
	\\
	Получим телескопическую сумму: $\frac{1}{4}(1 - \frac{1}{5} + \frac{1}{5} - \frac{1}{9} + ... - \frac{1}{4n  + 1}) = \frac{1}{4}(1 -  \frac{1}{4n  + 1})$
	\\
	\\
	\answer{$A_n = \frac{1}{4}(1 -  \frac{1}{4n  + 1})$} \bspace
	б) $B_n = \frac{1}{2} + \frac{3}{2 ^ 2} + \frac{5}{2 ^ 3} + \cdots + \frac{2n - 1}{2^{n}}$\\
	\\
	Умножим обе части уравнения на 2 и отделим из каждой дроби часть, которая входит в $B_n$. Также добавим и вычтем последний элемент из $B_n$, чтобы можно было упростить выражение\\
	\\
	$2B_n = 1  + \frac{3}{2} + \frac{5}{4}	+ \cdots + \frac{2n - 1}{2^{n - 1}}	= 1 +( \frac{1}{2} )+ \frac{2}{2} + (\frac{3}{2^2}) + \frac{2}{2^2} +( \frac{5}{2^3}) + \frac{2}{2^3} + \cdots + (\frac{2n - 3}{2^{n - 1}}) + \frac{2}{2^{n - 1}} + (\frac{2n - 1}{2^n}) - \frac{2n - 1}{2^n} = \\
	= B_n + 1 + \frac{2}{2} + \frac{2}{2^2} + \cdots + \frac{2}{2^{n - 1}} - \frac{2n - 1}{2^n}$\\\\
	Выразим $B_n$\\
	\\
	$B_n = 1 + 1 + \frac{1}{2} + \frac{1}{4} + \cdots + \frac{1}{2^{n - 2}} - \frac{2n - 1}{2^n}$\\
	\\
	Используем равенство $\frac{1}{2^1} + \frac{1}{2^2} + \cdots + \frac{1}{2 ^ k} = 1 - \frac{1}{2 ^ k}$, чтобы упростить выражение. \\
	\\
	$B_n = 1 + 1 + 1 - \frac{1}{2^{n - 2}} - \frac{2n - 1}{2^n} = 3 - \frac{2n + 3}{2^n}$\\\\
	\answer{$B_n = 3 - \frac{2n + 3}{2^n}$}
	\bspace
	в) $C_n = \frac{1}{3} + \frac{2}{3 ^ 2} + \frac{3}{3 ^3 } + \cdots + \frac{n}{3^ n}$\\
	\\
	Пусть $f(x) = x + x ^ 2 + x ^ 3 + \cdots + x ^ n$, тогда $f'(x) = 1 + 2x  + \cdots + nx^{n - 1}$\\\\
	Заметим, что $\frac{1}{3} * f'(\frac{1}{3}) = C_n$. Для того, чтобы вычислить $C_n$ сначала вычислим $f(x)$, затем посчитаем производную $f'(x)$ и подставим $f'(\frac{1}{3})$ в уравнение с $C_n$.\\
	\\
	$f(x)$ - сумма геометрической прогрессии $B$, где $b_1 = x$, $q = x$ вычислим сумму по формуле $S_n = \frac{b_1 (q^n - 1)}{q - 1}$\\
	\\
	$f(x) = \frac{x^{n + 1} - x}{x - 1}$\\
	\\
	$f'(x) = \frac{((n + 1)x^{n} - 1)*(x - 1) - (x^{n + 1} - x)}{(x - 1) ^ 2} = \frac{(nx^n + x^n - 1)(x - 1) - x^{n + 1} + x}{(x - 1)^2} = \frac{nx^{n + 1} + x ^ {n + 1} - x -nx^n - x^n + 1 - x^{n + 1} + x}{(x - 1) ^ 2} = \\
	= \frac{nx^{n + 1} - nx^n - x^n + 1}{(x - 1) ^ 2}$\\
	\\
	$f'(\frac{1}{3}) = \frac{\frac{n}{3 ^ {n + 1}} - \frac{n + 1}{3 ^ n}  + 1}{\frac{4}{9}} = \frac{9 + \frac{n}{3^{n - 1}} - \frac{n + 1}{3 ^ {n - 2}}}{4}$
	\\
	\\
	$C_n = \frac{1}{3} * f'(\frac{1}{3}) = \frac{3 + \frac{n}{3^{n}} - \frac{n + 1}{3 ^ {n - 1}}}{4}$ \\\\
	\answer{$C_n  = \frac{3 + \frac{n}{3^{n}} - \frac{n + 1}{3 ^ {n - 1}}}{4}$} \bspace
	\B{3. Доказать утверждения, применяя метод математической индукции}\\\\
	a) $\forall n \in \mathbb{N}$ верно: $1^ 2  + 2 ^ 2 + \cdots + n ^ 2 = \frac{n(n + 1)(2n + 1)}{6}$ \\
	\\
	\underline{База индукции $n = 1$:} $1 ^ 2 = \frac{1(1 + 1)(2 + 1)}{6}$\\
	\\ 
	$1  = \frac{6}{6}$ - верно\\
	\\
	\underline{Шаг индукции:} $1 ^ 2  + 2 ^ 2 + \cdots + n ^ 2 + (n + 1) ^ 2 = \frac{(n + 1)(n + 2)(2n + 3)}{6}$\\
	\\
	Используя предположение из условия заменим первые $n$ cлагаемых левой части уравнения \\
	\\
	$\frac{n(n + 1)(2n + 1)}{6} + (n  + 1) ^ 2 = \frac{(n + 1)(n + 2)(2n + 3)}{6}$\\
	\\
	$(n  + 1) ^ 2 =  \frac{(n + 1)(n + 2)(2n + 3) - n(n + 1)(2n + 1)}{6}$\\
	\\
	Сократим обе части уравнения на $n + 1 \neq 0$, тк $n > 0$\\
	\\
	$n + 1 =  \frac{(n + 2)(2n + 3) - n(2n  + 1)}{6}$\\
	\\
	Умножим обе части уравнения на 6\\
	\\
	$6n + 6 = 2n^2 + 7n + 6 - 2n^2 - n$\\
	\\
	$6n + 6 = 6n + 6$ - верно\\
	\\
	Так как база и шаг индукции верны, то $\forall n \in \mathbb{N}$ верно: $1^ 2  + 2 ^ 2 + \cdots + n ^ 2 = \frac{n(n + 1)(2n + 1)}{6}$\bspace
	б) Доказать, что \\\\ $\forall n: n \in \mathbb{N},  \forall \phi: \phi ^ 2 - \phi - 1 = 0,  F_k = \left \{ \begin{array}{l} 0, k = 0 \\ 1, k = 1 \\ F_{k - 1} + F_{k - 2} \end{array} \right .$ \\ выполняется $\phi ^ n = \phi * F_n + F_{n - 1}$\\
	\\
	\underline{База индукции $n = 1$:} $\phi = \phi * F_1 + F_0$\\
	$\phi = \phi * 1 +  0$ - верно\\
	\\
	\underline{Шаг индукции:} $\phi ^ {n + 1}= \phi * F_{n + 1} + F_{n}$\\
	\\
	$\phi * \phi^{n} = \phi *( F_n  + F{n - 1}) + F_n $\\
	\\
	Используя предположение заменим $\phi^n$ в левой части\\
	\\
	$\phi * (\phi * F_n + F_{n - 1}) = \phi *( F_n  + F_{n - 1}) + F_n $\\
	\\
	$\phi^2 * F_n + \phi * F_{n - 1} = \phi * F_n  + \phi * F_{n - 1} + F_n $\\
	\\
	$\phi^2 * F_n = F_n(\phi + 1)$\\
	\\
	$F_n(\phi^ 2 - \phi  - 1) = 0$
	Используем условие для $\phi: \phi ^ 2 - \phi - 1 = 0$\\
	\\
	$F_n * 0 = 0$ - верно\\
	\\
	Так как база и шаг индукции верны, то исходное утверждение верно.\\
	\\
	в) Доказать, что $\forall n \in \mathbb{N}: F_n < 2 ^ n$\\\\
	(в этом подпункте я пользуюсь методом полной математической индукции)
	\\\\
	Отдельно докажем, что для $n = 1$ условие выполняется\\
	\\
	$F_1 < 2 ^ 1$\\
	\\
	$1 < 2$ - верно\\
	\\
	(1) Теперь докажем верность исходного высказывания для n > 1\\
	\\
	\underline{База индукции $n = 2$:} $F_2 < 2 ^ 2$\\
	\\
	$1 < 4$ - верно\\
	\\
	\underline{Шаг индукции:} $F_{n + 1} < 2^{n + 1}$\\
	\\
	Для любого числа Фибоначчи верно $F_{n + 1} = F_{n} + F_{n - 1}$, заменим $F_{n + 1}$ в левой части уравнения\\
	\\
	$F_{n} + F_{n + 1} < 2^{n + 1}$\\
	\\
	Используя предположение для $F_{n}$ и $F_{n - 1}$ оценим $F_{n} + F_{n - 1} < 2 ^ n + 2^{n - 1}$\\
	\\
	Так как по оценке $ 2 ^ n + 2^{n - 1} > F_n  + F_{n + 1}$, то мы можем заменить левую часть на $ 2 ^ n + 2^{n - 1}$\\
	\\
	$ 2 ^ n + 2^{n - 1} < 2 ^ {n + 1}$\\
	\\
	$2 * 2^{n - 1} + 2^{n - 1} < 2 * 2 * 2^{n - 1}$\\
	\\
	$3 * 2^{n - 1} < 4 * 2^{n - 1}$\\
	\\
	Так как n > 1 по условию (1), то $2^{n - 1} \neq 0$, сократим обе части уравнения на $2^{n - 1}$\\
	\\
	$3 < 4$ - верно\\
	\\
	Так как база индукции и шаг индукции верны, то $\forall n \in \mathbb{N}: n > 1$ исходное высказывание верно.\\
	\\
	Так как для $n = 1$  и  $\forall n \in \mathbb{N}: n > 1$ высказывание верно, то верно $\forall n \in \mathbb{N}: F_n < 2 ^ n$
\end{document}