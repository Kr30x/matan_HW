\documentclass[11pt]{article}
\usepackage{gset}
\usepackage[]{preamble}
\usepackage{amsmath}

\begin{document}
	
	\section*{Список определений:}
	
	\subsection*{1. Логические связки, их таблицы истинности:}
	- \textbf{Объяснение:} Логические связки, такие как конъюнкция (\(\land\)), дизъюнкция (\(\lor\)), и импликация (\(\rightarrow\)), представляют собой операции над высказываниями.
	- \textbf{Пример:} Таблица истинности для конъюнкции:
	\[
	\begin{array}{ccc}
		P & Q & P \land Q \\
		\hline
		T & T & T \\
		T & F & F \\
		F & T & F \\
		F & F & F \\
	\end{array}
	\]
	
	\subsection*{2. Тавтологии и равносильные формулы:}
	- \textbf{Объяснение:} Тавтология - это истинное высказывание в любых значениях переменных. Равносильные формулы имеют одинаковое значение при всех значениях переменных.
	- \textbf{Пример:} \( (P \rightarrow Q) \equiv (\neg P \lor Q) \) - это тавтология.
	
	\subsection*{3. Доказательства разбором случаев. Доказательства от противного:}
	- \textbf{Объяснение:} Доказательство разбором случаев разделяет анализ на несколько частных случаев. Доказательство от противного предполагает ложность утверждения и выводит противоречие.
	- \textbf{Пример:} Доказательство от противного, что для любого вещественного числа \( x \), если \( x^2 \) чётно, то \( x \) тоже чётно.
	
	\subsection*{4. Равные множества. Подмножество. Пустое множество:}
	- \textbf{Объяснение:} Множества равны, если содержат одни и те же элементы. Подмножество - множество, все элементы которого также принадлежат другому множеству.
	- \textbf{Пример:} Если \( A = \{1, 2, 3\} \) и \( B = \{3, 2, 1\} \), то \( A = B \).
	
	\subsection*{5. Операции над множествами: объединение, пересечение, разность, симметрическая разность:}
	- \textbf{Объяснение:} Операции, применяемые к множествам для получения новых множеств.
	- \textbf{Пример:} Если \( A = \{1, 2, 3\} \) и \( B = \{2, 3, 4\} \), то \( A \cup B = \{1, 2, 3, 4\} \).
	
	\subsection*{6. Теоретико-множественные тождества:}
	- \textbf{Объяснение:} Идентичности, справедливые для множеств.
	- \textbf{Пример:} Дистрибутивность пересечения относительно объединения: \( A \cap (B \cup C) = (A \cap B) \cup (A \cap C) \).
	
\end{document}
