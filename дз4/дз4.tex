\documentclass[11pt]{article}
\usepackage{gset}
\usepackage{preamble}
\begin{document}
	\maintitle{Матан Дз 4}
	\textbf{1)}
	Пусть $a_n = \dfrac{1}{\sqrt{1}} + \dfrac{1}{\sqrt{2}} + \cdots + \dfrac{1}{\sqrt{n}} - 2\sqrt{n}, \quad b_n = \dfrac{1}{\sqrt{1}} + \dfrac{1}{\sqrt{2}} + \cdots + \dfrac{1}{\sqrt{n}} - 2\sqrt{n + 1}$ \bs
	a) $a_n - b_n = \dfrac{1}{\sqrt{1}} + \dfrac{1}{\sqrt{2}} + \cdots + \dfrac{1}{\sqrt{n}} - 2\sqrt{n} - \dfrac{1}{\sqrt{1}} - \dfrac{1}{\sqrt{2}} - \cdots - \dfrac{1}{\sqrt{n}} + 2\sqrt{n + 1} = 2\sqrt{n + 1} - 2\sqrt{n} > 0$,  \sspace потому что функция $f(x) = \sqrt{x}$ монотонно возрастает. \sspace
	b) 
	$a_n - a_{n + 1} = \dfrac{1}{\sqrt{1}} + \dfrac{1}{\sqrt{2}} + \cdots + \dfrac{1}{\sqrt{n}} - 2\sqrt{n} - \dfrac{1}{\sqrt{1}} - \dfrac{1}{\sqrt{2}} - \cdots - \dfrac{1}{\sqrt{n}} - \dfrac{1}{\sqrt{n + 1}} + 2\sqrt{n + 1} = 2\sqrt{n + 1} - 2\sqrt{n} -  \dfrac{1}{\sqrt{n + 1}} = \dfrac{2(n + 1) - 2\sqrt{n^2 + n} - 1}{\sqrt{n + 1}} = \dfrac{2n + 1 - 2\sqrt{n^2 + n}}{\sqrt{n + 1}} = \dfrac{\sqrt{4n^2 + 4n + 1} - \sqrt{4n^2 + 4n}}{\sqrt{n + 1}} > 0$\sspace значит последовательность монотонно убывает. \sspace
	c) 
	$b_{n + 1} - b_{n} = \dfrac{1}{\sqrt{1}} + \dfrac{1}{\sqrt{2}} + \cdots + \dfrac{1}{\sqrt{n}} + \dfrac{1}{\sqrt{n + 1}}- 2\sqrt{n + 2} - \dfrac{1}{\sqrt{1}} - \dfrac{1}{\sqrt{2}} - \cdots - \dfrac{1}{\sqrt{n}} + 2\sqrt{n + 1} = \sspace = \dfrac{1}{\sqrt{n + 1}} - 2\sqrt{n + 2} + 2\sqrt{n + 1} =\dfrac{1 + 2(n + 1) - 2\sqrt{(n + 2)(n + 1)}}{\sqrt{n + 1}}= \sspace = \dfrac{\sqrt{4n^2 + 12n + 9} - \sqrt{4n^2 + 12n + 8}}{\sqrt{n + 1}} > 0$ значит последовательность возрастает \sspace
	d) Пусть k > m:
	$a_k < a_m, b_k > b_m$. Пусть неверно $a_k > b_m$, тогда $a_k \leq b_m < b_k \Rightarrow a_k < b_k$ \sspace - противоречие с пунктом а. \sspace
	Пусть k < m:
	$a_k > a_m, b_k < b_m$. Пусть неверно $a_k > b_m$, тогда $a_m < a_k \leq b_m \Rightarrow a_m < b_m$ \sspace - противоречие с пунктом а. \sspace
	Пусть k = m: по пункту (а) $a_k > b_k$ \sspace
	e)
	$a_1 > a_2 > \cdots > a_n > b_1$, значит $a_n$ ограничена снизу. \\
	$b_1 < b_2 < \cdots < b_n < a_1$, значит $b_n$ ограничена сверху. \sspace
	f)
	Пусть $\lim\limits_{n\to \infty} a_n = A, \lim\limits_{n\to \infty} b_n  = B, \lim\limits_{n\to \infty} a_n - b_n = \lim\limits_{n\to \infty} (2\sqrt{n + 1} - 2\sqrt{n}) = 0$ \sspace
	$A - B = 0 \Rightarrow A = B$ \bs
	\textbf{2)} \sspace
	a) $a_n = 2 + (-1)^n$ \sspace
	$[C]\lim\limits_{n\to \infty} a_n = \lim\limits_{n\to \infty} \dfrac{a_1 + a_2 + \cdots + a_n}{n}$ \sspace
	Для четного n: $\lim\limits_{n\to \infty} \dfrac{a_1 + a_2 + \cdots + a_n}{n} = \dfrac{2 * n}{n} = 2$ \sspace
	Для нечетного n: $\lim\limits_{n\to \infty} \dfrac{a_1 + a_2 + \cdots + a_n}{n} = \lim\limits_{n\to \infty} \dfrac{2 * n - 1}{n} = 2$ \sspace
	\answer{2} \sspace
	b) $a_n = (1 + \dfrac{1}{n})^n$ \sspace
	$[C]\lim\limits_{n\to \infty} a_n = \lim\limits_{n\to \infty} \dfrac{a_1 + a_2 + \cdots + a_n}{n}$ \sspace
	Из семинарской задачи 3.1 известно, что последовательность $\lim\limits_{n\to \infty} a_n \eqdef e$. Так же по семинарской задаче 3.3 известно, что любая последовательность, которая имеет предел в обычном понимании, имеет такой же предел по Чезаро, значит $[C] \lim\limits_{n\to \infty} a_n = \lim\limits_{n\to \infty} a_n = e$ \sspace
	c) $a_n = \sin n$ \sspace 
	$S_n = \sin1 + \sin2 +\sin3 + \cdots + \sin n \sspace
	2\sin{\frac{1}{2}}*S_n = 2\sin{\frac{1}{2}}(\sin1 + \sin2 + \cdots + \sin n) \sspace
	2\sin{\frac{1}{2}}*S_n = \cos{\frac{1}{2}} - \cos{\frac{3}{2}} + \cos{\frac{3}{2}} - \cos{\frac{5}{2}} + \cdots + \cos{\frac{2n - 1}{2}} - \cos{\frac{2n + 1}{2}} \sspace
	2\sin{\frac{1}{2}}*S_n = \cos{\frac{1}{2}} - \cos{\frac{2n + 1}{2}} \sspace
	2\sin{\frac{1}{2}}*S_n = 2\sin{\frac{1 + 2n + 1}{2}}\sin{\frac{1 - 2n - 1}{2}} \sspace
	S_n = \dfrac{\sin{(n + 1)}\sin{n}}{\sin{\frac{1}{2}}} \sspace $
	$ [C]\lim\limits_{n\to \infty} a_n = \lim\limits_{n\to \infty} \dfrac{S_n}{n} = \lim\limits_{n\to \infty} \dfrac{\sin{(n + 1)}\sin{n}}{\sin{\frac{1}{2}} * n}  = 0$ \sspace
	\answer{0}
	\bs
	\textbf{3)}
	\bs
	a) $\lim\limits_{n\to \infty} \dfrac{1}{\sqrt{n}}\bigg(1 + \dfrac{1}{\sqrt{2}} + \cdots + \dfrac{1}{\sqrt{n}}\bigg)$ \bs
	Пусть $\{a_n\}_{n = 1}^\infty = \bigg(1 + \dfrac{1}{\sqrt{2}} + \cdots + \dfrac{1}{\sqrt{n}}\bigg)$, $\{b_n\}_{n = 1}^\infty = \sqrt{n}$. \sspace
	$\lim\limits_{n\to \infty} \dfrac{a_{n + 1} - a_n}{b_{n + 1} - b_n} = \lim\limits_{n\to \infty} \dfrac{\dfrac{1}{\sqrt{n + 1}}}{\sqrt{n + 1} - \sqrt{n}} = \lim\limits_{n\to \infty} \dfrac{\sqrt{n + 1} + \sqrt{n}}{\sqrt{n + 1}(n + 1 - n)} = \lim\limits_{n\to \infty} 1 + \dfrac{\sqrt{n}}{\sqrt{n + 1}} = 2$ \bs
	$b_n > 0 \quad b_{n - 1} - b_n > 0 \quad \lim\limits_{n\to \infty} b_n = \infty$ \sspace
	По теореме Штольца $\lim\limits_{n\to \infty} \dfrac{1}{\sqrt{n}}\bigg(1 + \dfrac{1}{\sqrt{2}} + \cdots + \dfrac{1}{\sqrt{n}}\bigg) = 2$ \sspace
	\answer{2} \bs
	b) $\lim\limits_{n\to \infty} \dfrac{1^k + 2^k + \cdots + n^k}{n^{k + 1}}$ \sspace
	Пусть $\{a_n\}_{n = 1}^\infty = 1^k + 2^k + \cdots + n^k$, $\{b_n\}_{n = 1}^\infty = n^{k + 1}$. \sspace
	$\lim\limits_{n\to \infty} \dfrac{a_{n + 1} - a_n}{b_{n + 1} - b_n} = \lim\limits_{n\to \infty} \dfrac{(n + 1) ^ k}{(n + 1) ^ {k + 1} - n^{k + 1}} = \bs = \lim\limits_{n\to \infty} \dfrac{n^k(1 + \dfrac{k}{n} + \dfrac{\lambda_1}{n^2} + \dfrac{\lambda_2}{n^3} + \cdots + \dfrac{\lambda_{k - 1}}{n^{k - 2}}+ \dfrac{k}{n^{k-1}} + \dfrac{1}{n^k})}{-n^{k + 1} + n^{k + 1} + n^k(k + 1 + \dfrac{\gamma_1}{n} + \dfrac{\gamma_2}{n^2} + \cdots + \dfrac{k + 1}{n^{k-1}} + \dfrac{1}{n^{k}})} = \dfrac{1}{k + 1}$ \bs
	Если словами объяснять эту страшную штуку, то $n^{k + 1}$ в знаменателе сократилась, поэтому старшая степень $n^k$, ее и выносим за скобки как в числителе, так и в знаменателе. В числителе при $n^k$ коёффициент 1, в знаменателе k + 1, это можно понять из треугольника Паскаля. Так как все остальные члены стремятся к 0, то можем брать в рассчет только отношение коэффициентов при старшей степени. \bs
	$b_n > 0 \quad b_{n - 1} - b_n > 0 \quad \lim\limits_{n\to \infty} b_n = \infty$ \sspace
	По теореме Штольца $\lim\limits_{n\to \infty} \dfrac{1^k + 2^k + \cdots + n^k}{n^{k + 1}} = \dfrac{1}{k + 1}$ \bs
	\answer{$\dfrac{1}{k + 1}$}
\end{document}